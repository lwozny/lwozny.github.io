% Options for packages loaded elsewhere
\PassOptionsToPackage{unicode}{hyperref}
\PassOptionsToPackage{hyphens}{url}
%
\documentclass[
]{article}
\usepackage{lmodern}
\usepackage{amssymb,amsmath}
\usepackage{ifxetex,ifluatex}
\ifnum 0\ifxetex 1\fi\ifluatex 1\fi=0 % if pdftex
  \usepackage[T1]{fontenc}
  \usepackage[utf8]{inputenc}
  \usepackage{textcomp} % provide euro and other symbols
\else % if luatex or xetex
  \usepackage{unicode-math}
  \defaultfontfeatures{Scale=MatchLowercase}
  \defaultfontfeatures[\rmfamily]{Ligatures=TeX,Scale=1}
\fi
% Use upquote if available, for straight quotes in verbatim environments
\IfFileExists{upquote.sty}{\usepackage{upquote}}{}
\IfFileExists{microtype.sty}{% use microtype if available
  \usepackage[]{microtype}
  \UseMicrotypeSet[protrusion]{basicmath} % disable protrusion for tt fonts
}{}
\makeatletter
\@ifundefined{KOMAClassName}{% if non-KOMA class
  \IfFileExists{parskip.sty}{%
    \usepackage{parskip}
  }{% else
    \setlength{\parindent}{0pt}
    \setlength{\parskip}{6pt plus 2pt minus 1pt}}
}{% if KOMA class
  \KOMAoptions{parskip=half}}
\makeatother
\usepackage{xcolor}
\IfFileExists{xurl.sty}{\usepackage{xurl}}{} % add URL line breaks if available
\IfFileExists{bookmark.sty}{\usepackage{bookmark}}{\usepackage{hyperref}}
\hypersetup{
  pdftitle={Egzamin z Mikroekonomii II},
  pdfauthor={prof. Łukasz Woźny},
  hidelinks,
  pdfcreator={LaTeX via pandoc}}
\urlstyle{same} % disable monospaced font for URLs
\usepackage[margin=1in]{geometry}
\usepackage{graphicx}
\makeatletter
\def\maxwidth{\ifdim\Gin@nat@width>\linewidth\linewidth\else\Gin@nat@width\fi}
\def\maxheight{\ifdim\Gin@nat@height>\textheight\textheight\else\Gin@nat@height\fi}
\makeatother
% Scale images if necessary, so that they will not overflow the page
% margins by default, and it is still possible to overwrite the defaults
% using explicit options in \includegraphics[width, height, ...]{}
\setkeys{Gin}{width=\maxwidth,height=\maxheight,keepaspectratio}
% Set default figure placement to htbp
\makeatletter
\def\fps@figure{htbp}
\makeatother
\setlength{\emergencystretch}{3em} % prevent overfull lines
\providecommand{\tightlist}{%
  \setlength{\itemsep}{0pt}\setlength{\parskip}{0pt}}
\setcounter{secnumdepth}{-\maxdimen} % remove section numbering
\usepackage{amsfonts}

\usepackage{graphicx}
\usepackage{epstopdf}
\usepackage{hyperref}
\usepackage{float}
\usepackage{geometry}
\usepackage{array}
\usepackage{natbib}
\usepackage{multirow}
\usepackage{amsmath}
\usepackage{amssymb}
\usepackage{graphicx,epstopdf}
\usepackage{setspace}
\usepackage{threeparttable}
\usepackage{longtable}
\usepackage{listings}
\usepackage{lscape}

\title{Egzamin z Mikroekonomii II}
\author{prof. Łukasz Woźny}
\date{18/06/2020}

\begin{document}
\maketitle

Czas na rozwiązanie zadań to 60 minut.\\
Proszę przesłać skany rozwiazań do swojego ćwiczeniowca do godziny
12:50.\\
Artur Krawczyk:
\href{mailto:ak56589@student.sgh.waw.pl}{\nolinkurl{ak56589@student.sgh.waw.pl}}\\
Przemysław Siemaszko:
\href{mailto:ps50943@doktorant.sgh.waw.pl}{\nolinkurl{ps50943@doktorant.sgh.waw.pl}}\\
W temacie pracy proszę podać słowo `egzamin'.

\subsection*{Zadanie 1. [5 pkt.]}

Na przykładzie preferencji doskonale substycyjnych graficznie przedstaw
działanie efektu dochodowego i substytucyjnego. Użyj dekompozycji
Hicksa.

\subsection*{Zadanie 2. [20 pkt.]}

W tym zadaniu przeanalizujesz międzyokresowy wybór konsumenta żyjącego
dwa okresy. Załózmy, ze w pierwszym okresie konsument posiada majątek w
wysokości \(w\), który moze przeznaczyć na konsumpcję \((c_1)\) i
oszczędności \((s)\). W drugim okresie jego majątek jest równy
oszczędnościom poczynionym w pierwszym okresie, powiększonym o stałą
stopę procentową \(r\), który w całosci jest konsumowany. Użyteczność
konsumenta ma postać \(u(c_1,c_2)=(\alpha_1c_1)(\alpha_2c_2)^\delta\),
gdzie \(c_1\), \(c_2\) oznaczają odpowiednio poziom konsumpcji w
pierwszym i drugim okresie.

\begin{itemize}
\item[(i)] Zapisz problem konsumenta maksymalizującego użyteczność w całym życiu. Zapisz odpowiadającą mu funkcję Lagrange'a.
\item[(ii)] Rozwiąż problem, okreslając optymalne poziomy konsumpcji $(c_1, c_2)$ i oszczędności $(s)$.
\item[(iii)] Jakiego rodzaju dobrami jest konsumpcja w pierwszym i drugim okresie? Czym w tym przypadku jest stopa procentowa $r$? Jak od niej zależy decyzja odnosnie konsumpcji w obydwu okresach?
\end{itemize}

\subsection*{Zadanie 3. [7 pkt.]}

Wyznacz cenę i produkcję monopolisty z funkcją kosztów \(TC(q)=cq\) oraz
popytem \(D(p)=Ap^{-b}\).

\subsection*{Zadanie 4. [8 pkt.]}

Narysuj macierz 2x2 z wypłatami przedstawiającymi grę z
komplementarnością strategii. Znajdź wszystkie równowagi Nasha.

\subsection*{Zadanie 5. [20 pkt.]}

Rozpatrujemy wyspę dwóch plemion Westerner i Easterners, które spotykają
się raz w roku na targu. Na wyspie mieszka 1000 rodzin Westerners, z
których każda produkuje 30 sztuk X-angów i 200 kwintali Y-amp. Wszystkie
mają preferencje określone przez funkcję użyteczności o wartosciach
\(U_w=X_w^{\frac{1}{2}}Y_w^{\frac{1}{2}}\). Jednocześnie na wyspie
mieszka 2000 rodzin Easterners, z których każda produkuje 25 X-angów i
300 kwintali Y-amp. Wszystkie mają preferencje określone przez funkcję
użyteczności o wartościach \(U_e=X_e^{\frac{3}{4}} Y_e^{\frac{1}{4}}\).
Na wyspie targ działa na zasadzie konkurencji Arrow-Debreu.

\begin{itemize}
\item Jaki jest stosunek cen Y-amp do X-angów?
\item Ile każdego z dóbr konsumuje każda z 2 typów rodzin na
wyspie?
\end{itemize}

\subsection*{Zadanie 6. [10 pkt.]}

Podaj przykład reprezentujący paradoks Alais. Dlaczego nie spełnia on
aksjomatów von Neumanna-Morgersterna?

\end{document}
